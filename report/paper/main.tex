
\documentclass{article} % For LaTeX2e
\usepackage{template/iclr2026_template/iclr2026_conference}
\usepackage{times}

% Optional math commands from https://github.com/goodfeli/dlbook_notation.
\input{math_commands.tex}

\usepackage{hyperref}
\usepackage{url}


\title{Complex-Valued CNNs for fastMRI Knee Reconstruction: A Tiny Reproduction}

% Authors must not appear in the submitted version. They should be hidden
% as long as the \iclrfinalcopy macro remains commented out below.
% Non-anonymous submissions will be rejected without review.

\author{Anonymous Authors \\
Anonymous Institution \\
\texttt{anonymous@tinyreproductions.edu}}

\newcommand{\fix}{\marginpar{FIX}}
\newcommand{\new}{\marginpar{NEW}}

%\iclrfinalcopy % Uncomment for camera-ready version, but NOT for submission.
\begin{document}


\maketitle

% Outline content placeholder until sections are populated.

\begin{abstract}
\textbf{Motivation.} Rapid MRI requires aggressive undersampling, but zero-filled reconstructions yield ringing and blur. \textbf{Goal.} Reproduce the key claim of \citet{cole2020analysisdeepcomplexvaluedconvolutional}: capacity-matched complex-valued CNNs outperform real-valued CNNs on accelerated single-coil knee data. \textbf{Approach.} fastMRI single-coil knee dataset, equispaced undersampling with ACS, complex U-Net vs. real U-Net, identical training/evaluation protocols. \textbf{Findings.} Outline planned quantitative comparison (PSNR/SSIM/L1 deltas), qualitative improvements, and reproducibility contributions. (Will be replaced with final summary.)
\end{abstract}

\section{Introduction}
\subsection{Clinical Motivation}
    Magnetic Resonance Imaging (MRI) is a non-invasive imaging modality widely used for diagnosing musculoskeletal conditions. However, traditional MRI scans are time-consuming, often leading to patient discomfort and increased healthcare costs. Accelerated MRI techniques aim to reduce scan times by acquiring less data, but this introduces challenges in image reconstruction. Recent work \citep{schlemper2017deepcascadeconvolutionalneural} \citep{hammernik2017learningvariationalnetworkreconstruction} \citep{Aggarwal_2019} has explored the use of deep learning models, particularly convolutional neural networks (CNNs), to reconstruct high-quality images from undersampled k-space data. They do so by taking the full k-space data, simulating undersampling by applying a mask, reconstructing the image using both the complete and undersampled data, and then training a model to minimize the difference between the two reconstructions. With a sufficiently trained model, it becomes possible to significantly reduce MRI acquisition times while maintaining diagnostic image quality.
\subsection{Paper to Reproduce}
    Prior work in the field has primarily focused on real-valued CNNs when reconstructing MRI images from undersampled data. However, this information is inherently complex-valued, as MRI data consists of both magnitude and phase components. Researchers tackle this by either transforming complex data into two channels of real numbers \citep{schlemper2017deepcascadeconvolutionalneural} \citep{hammernik2017learningvariationalnetworkreconstruction}, or by taking only the magnitude \citep{auckly2017isotopysurfaces4manifoldssingle}. Consequentially, the two channel approach may fail to model interactions between real and imaginary components, losing potentially valuable information, while the magnitude-only approach discards phase information entirely. Complex-valued CNNs, on the other hand, can directly process complex numbers, potentially leading to improved reconstruction quality. \cite{cole2020analysisdeepcomplexvaluedconvolutional} investigate the efficacy of fully complex-valued CNNs for MRI reconstruction, directly comparing them to their parameter-matched real-valued counterparts. Their experiments span unrolled networks and U-nets; however this paper will zoom in on the U-net comparison.
\subsection{Reproduction Goals}
    The primary objective of this project is to reproduce the key findings of \cite{cole2020analysisdeepcomplexvaluedconvolutional}, specifically the claim that capacity-matched complex-valued U-Nets can outperform real-valued U-Nets in reconstructing undersampled MRI data. I focus on fastMRI \citep{zbontar2019fastmriopendatasetbenchmarks} single-coil knees at $R=6$ with 20 ACS lines, train real and complex U-Nets under identical seeds and hyperparameters, and compare PSNR/SSIM/$\ell_1$ alongside qualitative slices. The pipeline, configs, and logging mirror the experimental design of the original paper while remaining lightweight enough to run on a trimmed dataset.
\subsection{Contributions}
    This work contributes a reproducible, single-seed comparison of real versus complex U-Nets on fastMRI knees with equispaced masks: parameter-matched real and complex U-Nets are trained with shared preprocessing and training code, and quantitative and qualitative outputs saved as publication-ready assets (Table~\ref{tab:main_results}, Figures~\ref{fig:training_curve} and \ref{fig:qualitative}).

\section{Background and Related Work}
\subsection{MRI Reconstruction Primer}
    At a high level, MRI acquires data in the frequency domain (k-space) by measuring the response of hydrogen nuclei to magnetic fields. Specifically, it collects the complex-valued Fourier coefficients of the spatial image, which can then be transformed back into the image domain using an inverse Fast Fourier Transform (iFFT). In a perfect world, the k-space would be sampled at the Nyquist rate to ensure accurate reconstruction. However, this is often impractical due to time and resource constraints. Instead, undersampling strategies are employed to reduce acquisition time and costs at the expense of image quality. While there are various undersampling patterns (random, variable density, etc.), this work focuses on equispaced undersampling with an auto-calibration signal (ACS) region. This means the we sample every $R^{th}$ line in k-space, where $R$ is the acceleration factor, while preserving a small central block of low-frequency lines (the ACS) to aid in reconstruction. In general, ACS lines can be viewed as the crucial low-frequency information needed for reconstruction, as they capture the overall structure and contrast of the image. As such, they are sampled fully even when undersampling. As we increase the acceleration factor $R$, we sample fewer lines in k-space outside of the ACS region, leading to faster acquisitions but more challenging reconstruction for high frequency features such as edges and fine details. We aim to restore these details using deep learning models.
% \begin{itemize}
%     \item \textbf{Forward model recap.} MRI acquires Fourier coefficients in k-space; undersampling corresponds to element-wise masking of the fully sampled k-space tensor followed by an inverse FFT to get a zero-filled image.
%     \item \textbf{Role of ACS.} Auto-calibration signal (ACS) lines preserve the low-frequency content needed for sensitivity estimation or residual learning; we follow fastMRI convention with centered ACS of width 8--24 lines.
%     \item \textbf{Metrics we will report.} PSNR (signal fidelity), SSIM (structural similarity on magnitude images), and $\ell_1$ (perceptual sharpness proxy). These are computed after per-slice max normalization to mirror \citet{cole2020analysisdeepcomplexvaluedconvolutional}.
%     \item \textbf{Figure placeholder.} \textit{Figure~\ref{fig:pipeline_placeholder}} (to be added) will illustrate the mask $\rightarrow$ inverse FFT $\rightarrow$ reconstruction flow.
% \end{itemize}
\subsection{Complex-Valued Neural Networks}
    Complex-valued neural networks (CVNNs) extend traditional real-valued networks by working directly with complex numbers. At first glance, this may seem like a straightforward extension. However, CVNNs require careful consideration of operations such as convolution, activation functions, and normalization to ensure stable and effective training. This is mainly due to the fact that each parameter has two degrees of freedom (real and imaginary parts), and interactions between them can be non-trivial. Small adjustments in phase can compound throughout the network and lead to numerical instabilities if not handled properly; something I encountered in this project. While the field of CVNNs is fascinating, this section will focus primarily on the relevant components needed to reproduce \citet{cole2020analysisdeepcomplexvaluedconvolutional}'s complex U-Net. \cite{trabelsi2018deepcomplexnetworks} provide a comprehensive overview of CVNN modules, of which the following were implemented in this work.

    \subsubsection{Complex Convolutions}
    In real-valued CNNs, convolutional layers apply a set of learnable filters to the input data, producing feature maps that capture spatial information. In the spirit of abusing notation, we can denote this operation as $Y = W * X$ where $W$ is the filter tensor, $X$ is the input tensor, and $Y$ is the output tensor. In complex-valued CNNs, both the input and filters are complex-valued, meaning they have real and imaginary components. The complex convolution operation can be expressed as: $Y = W * X = (A + jB) * (X + jY) = (A * X - B * Y) + j(B * X + A * Y)$ where $W = A + jB$ and $X = X + jY$ \cite{trabelsi2018deepcomplexnetworks}. This effectively doubles the number of parameters compared to a real-valued convolution with the same kernel size and channel counts. 

    \subsubsection{Complex Activations}
    Activation functions introduce non-linearity into neural networks, enabling them to learn complex mappings. This is particularly challenging in CVNNs, as it can be shown that no non-linear function can be both complex-differentiable (holomorphic) and non-constant \cite{trabelsi2018deepcomplexnetworks}. Various strategies have been proposed to address this, such as applying real-valued activations separetely to the real and imaginary parts, or using magnitude-phase based activations. \cite{trabelsi2018deepcomplexnetworks} introduce a handful of complex activations, including CReLU (Complex ReLU), which can be formulated as $CReLU(z) = ReLU(Re(z)) + jReLU(Im(z))$. Another option is zReLU, which can be expressed as $zReLU(z) = z$ if $0 < arg(z) < \pi/2$ else $0$, effectively zeroing out any complex number not in the first quadrant. \cite{cole2020analysisdeepcomplexvaluedconvolutional} found that CReLU worked best for their complex U-Net, so I adopt it here.

    \subsubsection{Complex Batch Normalization}
    Batch Normalization, introduced by \cite{ioffe2015batchnormalizationacceleratingdeep} is a technique used to accelerate training and improve stability by normalizing layer inputs. It works by standardizing each channel to have zero mean and unit variance across the mini-batch, often followed by learnable scaling and shifting parameters. \cite{trabelsi2018deepcomplexnetworks} extend this concept to complex-valued inputs by treating each complex number as a 2D vector in the real-imaginary plane, and normalizing based on the covariance matrix of these vectors. This involves computing the mean vector and covariance matrix for each channel, then using these statistics to normalize the inputs. This is formulated as $\tilde{z} = V^{-\frac{1}{2}}(z-\mu)$ where $V$ is the covariance matrix and $\mu$ is the mean vector. This approach ensures that both the real and imaginary components are normalized in a way that preserves their joint distribution, which is crucial for maintaining the integrity of complex-valued features. As with classical batch norm, learnable scaling and shifting parameters $\gamma$ and $\beta$ are applied after normalization, where $\gamma=\begin{bmatrix}\gamma_{rr} & \gamma_{ri} \\ \gamma_{ir} & \gamma_{ii}\end{bmatrix}$ is a $2\times2$ matrix that can scale and rotate the normalized complex features, and $\beta$ is a complex bias term (in vector form). While \cite{cole2020analysisdeepcomplexvaluedconvolutional} does not explicitly state whether they used complex normalization, or which variant, I found that using complex batch norm from \cite{trabelsi2018deepcomplexnetworks} was crucial for training stability.

    \subsubsection{Initialization and Pooling Choices}
    To keep the variance of real and imaginary filters balanced, I use the complex Xavier-style initialization from \cite{trabelsi2018deepcomplexnetworks} (mirroring Glorot) for paired real/imaginary convolutional kernels; this matches the recommended practice in Cole et al.\ and avoids early-scale imbalances. The real-valued U-Net relies on PyTorch's default Kaiming initialization, which empirically keeps activation scales comparable. For downsampling, both networks use max-style pooling applied channelwise (component-wise pooling in the complex case) with $2\times2$ kernels and stride 2; this preserves the relative magnitudes of the real and imaginary components without introducing phase distortion that a magnitude-only pooling might cause.

% \begin{itemize}
%     \item \textbf{Complex convolutions.} Implemented as $W * x = (X * a - Y * b) + j(Y * a + X * b)$ where $a,b$ are real/imag parts; effectively doubles parameters compared to a real conv with the same channel counts.
%     \item \textbf{Activation + normalization.} C>ole et al. observed that simple CReLU activations and complex instance/batch norm are sufficient for U-Nets; we mirror their CReLU choice and rely on per-slice normalization in the dataset.
%     \item \textbf{Capacity matching.} Prior work matches the total trainable parameters when comparing complex vs. real nets by halving one network’s width; we keep this in mind when defining `features` and `width_scale` in the notebook.
%     \item \textbf{Debug considerations.} Complex nets are prone to numerical blow-ups if the learning rate is too high (see debug notebook results); hence we document the final LR used in Section~\ref{sec:method}.
% \end{itemize}
\subsection{Dataset: fastMRI Single-Coil Benchmarks}
    The fastMRI dataset, introduced by \cite{zbontar2019fastmriopendatasetbenchmarks} is a large-scale collection of MRI scans designed to facilitate research in accelerated MRI reconstruction. It includes both single-coil and multi-coil data, with various anatomies such as knee and brain scans. Crucially, the dataset provides fully sampled k-space data, allowing researchers to simulate undersampling and evaluate reconstruction algorithms. In this work, I focus on the single-coil knee subset, which consists of over 1,000 scans from different patients. Each scan features multiple slices, with each slice containing a 2D k-space tensor. These slices vary in spatial dimensions, so I center-crop them to a consistent size of $640\times320$ for training and evaluation. 
% \begin{itemize}
%     \item \textbf{Dataset quirks.} Single-coil HDF5 files contain complex-valued tensors with varying spatial sizes (e.g., $640\times368$); we center-crop to $640\times320$ for consistency.
%     \item \textbf{Baselines.} The fastMRI leaderboard reports zero-fill, classical compressed sensing, and real-valued U-Nets. Public complex-valued baselines are scarce, motivating this reproduction.
%     \item \textbf{Simplifications vs. Cole et al.} We use single-coil data, fewer epochs (15), and a single acceleration factor in the main comparison; this should be highlighted as a limitation later.
%     \item \textbf{Table placeholder.} \textit{Table~\ref{tab:dataset}} will summarize dataset splits, number of slices, and mask parameters used in our experiments.
% \end{itemize}

\section{Reproduction Target and Goals}
\label{sec:targets}
\subsection{Target from Cole et al.}
\cite{cole2020analysisdeepcomplexvaluedconvolutional} demonstrated that the complex-valued U-Net architecture outperformed its real-valued counterpart for various settings of the acceleration rate $R$, varying from $R=4$ to $R=9$. Due to time and resource constraints, I will focus on reproducing the $R=6$ results only, which is the middle of the range studied in the paper and should still be representative of the overall findings. Likewise, I will match the ACS lines to 20, as in the original paper. For this reproduction, I will focus on the fastMRI \cite{zbontar2019fastmriopendatasetbenchmarks} knee single-coil dataset only, as it is a reasonable subset of the most commonly used benchmark for MRI reconstruction tasks (fastMRI). I will train both the complex-valued U-Net and the real-valued U-Net baseline (with matching parameter counts) under identical conditions (dataset splits, hyperparameters, training epochs, random seeds) to compare their performance in terms of PSNR, SSIM, and L1 loss on the validation set.
\subsection{Acceptance Criteria}
To consider the reproduction successful, I will ensure that both real and complex models are trained and evaluated under identical seeds, hyperparameters, and dataset splits, and that the reported PSNR/SSIM/$\ell_1$ at $R=6$, ACS=20 are comparable to the trends reported by \citet{cole2020analysisdeepcomplexvaluedconvolutional}. Whether the complex model beats the real baseline or not, the outcome will be documented with matched capacity and transparent logging.
\subsection{Scope and Simplifications}
To keep the reproduction tractable within the constraints of time and computational resources, I only focus on the U-Net architecture and a single acceleration rate of $R=6$. I will not explore multi-coil data, adversarial losses, or alternate architectures. The experiments will be limited to a single pair of runs (real vs complex) with one random seed each, rather than multiple seeds for statistical significance. The details of these runs are covered in the subsequent sections.

\section{Methodology}
\subsection{Dataset and Preprocessing}
\begin{itemize}
    \item \textbf{Splits.} 1,024 training slices and 256 validation slices sampled from the official fastMRI single-coil train/val sets (indices captured in config files for reproducibility).
    \item \textbf{Preprocessing.} Load HDF5 k-space, apply equispaced mask with ACS=24 (offset fixed per slice), inverse FFT, add channel dimension, and center-crop to $640\times320$; normalize both masked and full images by the same slice-wise max magnitude (Section~\ref{sec:background}).
    \item \textbf{Debug hooks.} The `debug_trainer.ipynb` notebook mirrors this pipeline exactly for rapid inspection; mention in Methods to justify hyperparameter tweaks.
\end{itemize}
\subsection{Baselines}
\begin{itemize}
    \item \textbf{Zero-filled reconstruction.} Baseline pipeline for evaluating masks; metrics stored in `results/<timestamp>_zerofill/zero_fill_metrics.csv` and visualized in Figure~\ref{fig:zerofill_grid}.
    \item \textbf{Real U-Net.} Encoder/decoder widths $[64,128,256,512,1024]$ scaled by $1.416$ to match the target parameter count. Convs use ReLU, max-pooling in the encoder, bilinear interpolation safety via padding="same".
    \item \textbf{Complex U-Net.} Same macro-architecture as the real model but implemented with ComplexConv2d/ComplexConvTranspose2d and CReLU activations; widths $[64,128,256,512,1024]$.
\end{itemize}
\subsection{Complex-Valued U-Net}
\begin{itemize}
    \item \textbf{Layers.} ComplexConv2d pairs (real + imag kernels) plus optional pooling layers; skip connections are resized via per-component interpolation to handle odd shapes.
    \item \textbf{Activations.} CReLU applied to real/imag parts separately, as in \citet{cole2020analysisdeepcomplexvaluedconvolutional}.
    \item \textbf{Normalization.} We rely on dataset-level magnitude normalization rather than complex batch norm to keep the implementation lightweight.
\end{itemize}
\subsection{Training and Evaluation Protocol}
\begin{itemize}
    \item \textbf{Hardware.} Apple M2 (MPS) laptop; all runs seeded (Python/NumPy/PyTorch) for determinism.
    \item \textbf{Optimizer/schedule.} Adam with learning rate $5\times10^{-5}$ (after debugging), batch size 4, 15 epochs; no weight decay. Mention that a higher LR caused divergence for the complex net (cite debug appendix).
    \item \textbf{Loss/metrics.} Primary loss is $\ell_1$ on complex outputs; PSNR/SSIM/L1 reported each epoch and saved to `metrics.csv`.
    \item \textbf{Logging.} Each experiment writes results to `results/<timestamp>_<tag>/` with config snapshots, metrics CSV/JSON summaries, best/latest checkpoints, qualitative PNGs (Figure~\ref{fig:qualitative}), and optional tensor dumps for difference maps.
\end{itemize}
\end{itemize}

\section{Experiments}
\subsection{Zero-Filled Sanity Sweep}
\begin{itemize}
    \item Evaluate 64 validation slices across $R \in \{2,4,6,8\}$ and ACS $\in \{8,16,24\}$; store metrics in \texttt{results/<timestamp>\_zerofill/zero\_fill\_metrics.csv}.
    \item Recreate the grid from Cole et al.\ showing how larger ACS mitigates aliasing at each $R$ (placeholder Figure~\ref{fig:zerofill_grid}); this figure also doubles as a preprocessing sanity check.
    \item Talk about notable observations: PSNR drops monotonically with $R$, ACS width provides $\approx$2\,dB boost at each $R$.
\end{itemize}
\subsection{Real U-Net Baseline}
\begin{itemize}
    \item Train for 15 epochs (Adam, LR=$5\times10^{-5}$) on the 1,024/256 subset; log metrics per epoch and save best/latest checkpoints.
    \item Include a plot of training/validation curves (placeholder Figure~\ref{fig:training_curve}) to demonstrate stability after lowering the LR.
    \item Generate qualitative PNGs for consistent slice indices (e.g., index 50) to reuse later in the real vs.\ complex comparison.
\end{itemize}
\subsection{Complex U-Net Reproduction}
\begin{itemize}
    \item Repeat the exact training loop with the complex U-Net and note any differences (e.g., gradient explosions at LR=$1\times10^{-4}$, resolved by halving the LR); cite debug notebook in Appendix.
    \item Save per-epoch metrics and generate qualitative PNGs for the same slice as the real baseline; compute per-pixel difference heatmaps saved under \texttt{qualitative/} for later figure assembly.
    \item Table~\ref{tab:main_results} will summarize zero-fill vs.\ real vs.\ complex metrics; discuss whether the complex model hits the hypothesized PSNR/SSIM margins.
\end{itemize}
\subsection{Additional Ablations (if executed)}
\begin{itemize}
    \item Optional ideas (even if not run, mention as future work): alternate ACS widths at $R=4$, testing complex activations (modReLU/cardioid), or comparing different qualitative slice indices.
    \item Placeholder for a table/figure if time permits (e.g., Table~\ref{tab:ablation}).
\end{itemize}

\section{Results and Discussion}
\label{sec:results}
\label{sec:discussion}
\subsection{Results Summary}
Table~\ref{tab:main_results} reports PSNR/SSIM/$\ell_1$ for the zero-filled baseline, real U-Net, and complex U-Net at $R=6$, ACS=20 on the 1{,}024-slice validation set. The real U-Net reaches 25.10 dB PSNR with 0.785 SSIM; the complex U-Net edges ahead at 25.49 dB PSNR, 0.887 SSIM, and the lowest $\ell_1$ (0.0408), indicating a modest but consistent lift with matched capacity on 8{,}192 training slices.

\begin{table}[h]
\centering
\caption{Validation metrics at $R=6$, ACS=20 for zero-fill, real, and complex U-Nets (PSNR/SSIM/$\ell_1$).}
\label{tab:main_results}
\begin{tabular}{lccc}
\toprule
Model & PSNR (dB) & SSIM & $\ell_1$ \\
\midrule
Zero-fill & 21.78 & 0.878 & 0.0584 \\
Real U-Net & 25.10 & 0.785 & 0.0462 \\
Complex U-Net & \textbf{25.49} & \textbf{0.887} & \textbf{0.0408} \\
\bottomrule
\end{tabular}
\end{table}
\noindent Figure~\ref{fig:qualitative} shows matched slices for the two models and ground truth at a fixed validation index, with PSNR/SSIM overlaid to align visual and quantitative signals. Training curves in Figure~\ref{fig:training_curve} show epoch-level losses and PSNR to reveal any drift or divergence under the shared optimizer and schedule.

\begin{figure}[H]
\centering
\includegraphics[width=0.65\linewidth]{figures/fig_qualitative_idx_50_250_630.png}
\caption{Qualitative reconstructions at a fixed validation index (zero-fill, real U-Net, complex U-Net, ground truth) with PSNR/SSIM overlaid.}
\label{fig:qualitative}
\end{figure}

\begin{figure}[H]
\centering
\includegraphics[width=0.8\linewidth]{figures/fig_training_curve.png}
\caption{Training curves showing per-step losses (left) and validation PSNR per epoch (right) for real and complex U-Nets.}
\label{fig:training_curve}
\end{figure}

\subsection{Scope differences and limitations}
This reproduction is narrower than \citet{cole2020analysisdeepcomplexvaluedconvolutional}: I use single-coil fastMRI knees only (no multi-coil or multi-dataset setting), equispaced masks at a fixed $R=6$/ACS=20 instead of variable-density Poisson masks across accelerations, and ~50k steps on 8k/1k slices with a single seed. I also restrict the comparison to U-Nets (no unrolled networks or activation-function ablations) and report magnitude PSNR/SSIM/$\ell_1$ without phase-specific metrics. These choices likely explain why the complex model’s gain is modest relative to the larger improvements reported in the original paper. Future work includes extending to multi-coil data, broader acceleration ranges, phase-sensitive metrics, and multiple seeds/architectures.

\section{Limitations and Future Work}
Key limitations are summarized in Section~\ref{sec:results} (Scope differences vs.\ original). Future work remains the same: expand to multi-coil and additional datasets, explore other accelerations and activations, add phase-sensitive metrics, and run multiple seeds.

% TODO(Conclusion): Update claims once quantitative results are finalized; adjust if complex underperforms real baseline.
\section{Conclusion}
I implemented and trained parameter-matched real and complex U-Nets on a trimmed fastMRI knee subset, logging zero-fill baselines, training curves, and qualitative slices for reproducibility. The repo now contains scripts, configs, and publication-ready figures/tables that make the comparison transparent, regardless of whether the complex model ultimately outperforms the real baseline. Remaining work includes scaling to larger datasets, adding phase-aware metrics, and extending beyond U-Nets, but the core pipeline is in place for others to reproduce or extend these findings.


\bibliography{refs}
\bibliographystyle{template/iclr2026_template/iclr2026_conference}

\end{document}
