\section{Introduction}
% TODO(Intro): Replace placeholder citations, align contributions with actual outcomes (complex may underperform), and sync dataset/mask splits with Methodology/Experiments.
% TODO: Replace placeholder "(PROVIDE CITATIONS)/(INSERT CITATION)" with actual references, smooth prose into full paragraphs,
% explicitly mention fastMRI in motivation/goals, and note our scope differences (single-coil, fewer epochs) versus the original paper.
\subsection{Clinical Motivation}
    Magnetic Resonance Imaging (MRI) is a non-invasive imaging modality widely used for diagnosing musculoskeletal conditions. However, traditional MRI scans are time-consuming, often leading to patient discomfort and increased healthcare costs. Accelerated MRI techniques aim to reduce scan times by acquiring less data, but this introduces challenges in image reconstruction. Recent work (PROVIDE CITATIONS) has explored the use of deep learning models, particularly convolutional neural networks (CNNs), to reconstruct high-quality images from undersampled k-space data. They do so by taking the full k-space data, and simulating undersampling by applying a mask. They then reconstruct the image using both the complete and undersampled data, and train a model to minimize the difference between the two reconstructions. With a sufficiently trained model, it becomes possible to significantly reduce MRI acquisition times while maintaining diagnostic image quality.
\subsection{Paper to Reproduce}
    Prior work in the field has primarily focused on real-valued CNNs when reconstructing MRI images from undersampled data. However, this information is inherently complex-valued, as MRI data consists of both magnitude and phase components. Researchers (INSERT CITATION) tackle this by either transforming complex data into two channels of real numbers, or by taking only the magnitude. Consequentially, the two channel approach may fail to model interactions between real and imaginary components, losing potentially valuable information, while the magnitude-only approach discards phase information entirely. Complex-valued CNNs, on the other hand, can directly process complex numbers, potentially leading to improved reconstruction quality. \cite{cole2020analysisdeepcomplexvaluedconvolutional} investigate the efficacy of fully complex-valued CNNs for MRI reconstruction, directly comparing them to their parameter-matched real-valued counterparts. Their experiments span unrolled networks and U-nets; however this paper will zoom in on the U-net comparison.
\subsection{Reproduction Goals}
    The primary objective of this project is to reproduce the key findings of \cite{cole2020analysisdeepcomplexvaluedconvolutional}, specifically the claim that capacity-matched complex-valued U-Nets can outperform real-valued U-Nets in reconstructing undersampled MRI data. I focus on fastMRI single-coil knees at $R=6$ with 20 ACS lines, train real and complex U-Nets under identical seeds and hyperparameters, and compare PSNR/SSIM/$\ell_1$ alongside qualitative slices. The pipeline, configs, and logging mirror the experimental design of the original paper while remaining lightweight enough to run on a trimmed dataset.
\subsection{Contributions}
    This work contributes a reproducible, single-seed comparison of real versus complex U-Nets on fastMRI knees with equispaced masks, including: (i) a zero-fill baseline table (Table~\ref{tab:zerofill_metrics}) and grid (Figure~\ref{fig:zerofill_grid}) to anchor reconstruction quality; (ii) parameter-matched real and complex U-Nets with shared preprocessing and training code; (iii) quantitative and qualitative outputs saved as publication-ready assets (Table~\ref{tab:main_results}, Figures~\ref{fig:training_curve} and \ref{fig:qualitative}); and (iv) debugging hooks to make divergences and underperformance transparent. Placeholder citations will be resolved once the bibliography is finalized.
