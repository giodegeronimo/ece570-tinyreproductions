\section{Reproduction Target and Goals}
\label{sec:targets}
\subsection{Target from Cole et al.}
\cite{cole2020analysisdeepcomplexvaluedconvolutional} demonstrated that the complex-valued U-Net architecture outperformed its real-valued counterpart for various settings of the acceleration rate $R$, varying from $R=4$ to $R=9$. Due to time and resource constraints, I will focus on reproducing the $R=6$ results only, which is the middle of the range studied in the paper and should still be representative of the overall findings. Likewise, I will match the ACS lines to 20, as in the original paper. For this reproduction, I will focus on the fastMRI \cite{zbontar2019fastmriopendatasetbenchmarks} knee single-coil dataset only, as it is a reasonable subset of the most commonly used benchmark for MRI reconstruction tasks (fastMRI). I will train both the complex-valued U-Net and the real-valued U-Net baseline (with matching parameter counts) under identical conditions (dataset splits, hyperparameters, training epochs, random seeds) to compare their performance in terms of PSNR, SSIM, and L1 loss on the validation set.
\subsection{Acceptance Criteria}
To consider the reproduction successful, I will ensure that both real and complex models are trained and evaluated under identical seeds, hyperparameters, and dataset splits, and that the reported PSNR/SSIM/$\ell_1$ at $R=6$, ACS=20 are comparable to the trends reported by \citet{cole2020analysisdeepcomplexvaluedconvolutional}. Whether the complex model beats the real baseline or not, the outcome will be documented with matched capacity and transparent logging.
\subsection{Scope and Simplifications}
To keep the reproduction tractable within the constraints of time and computational resources, I only focus on the U-Net architecture and a single acceleration rate of $R=6$. I will not explore multi-coil data, adversarial losses, or alternate architectures. The experiments will be limited to a single pair of runs (real vs complex) with one random seed each, rather than multiple seeds for statistical significance. The details of these runs are covered in the subsequent sections.
