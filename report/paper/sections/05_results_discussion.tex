\section{Results and Discussion}
\label{sec:results}
\label{sec:discussion}
\subsection{Results Summary}
Table~\ref{tab:main_results} reports PSNR/SSIM/$\ell_1$ for the zero-filled baseline, real U-Net, and complex U-Net at $R=6$, ACS=20 on the 1{,}024-slice validation set. The real U-Net reaches 25.10 dB PSNR with 0.785 SSIM; the complex U-Net edges ahead at 25.49 dB PSNR, 0.887 SSIM, and the lowest $\ell_1$ (0.0408), indicating a modest but consistent lift with matched capacity on 8{,}192 training slices.

\begin{table}[h]
\centering
\caption{Validation metrics at $R=6$, ACS=20 for zero-fill, real, and complex U-Nets (PSNR/SSIM/$\ell_1$).}
\label{tab:main_results}
\begin{tabular}{lccc}
\toprule
Model & PSNR (dB) & SSIM & $\ell_1$ \\
\midrule
Zero-fill & 21.78 & 0.878 & 0.0584 \\
Real U-Net & 25.10 & 0.785 & 0.0462 \\
Complex U-Net & \textbf{25.49} & \textbf{0.887} & \textbf{0.0408} \\
\bottomrule
\end{tabular}
\end{table}
\noindent Figure~\ref{fig:qualitative} shows matched slices for the two models and ground truth at a fixed validation index, with PSNR/SSIM overlaid to align visual and quantitative signals. Training curves in Figure~\ref{fig:training_curve} show epoch-level losses and PSNR to reveal any drift or divergence under the shared optimizer and schedule.

\begin{figure}[H]
\centering
\includegraphics[width=0.65\linewidth]{figures/fig_qualitative_idx_50_250_630.png}
\caption{Qualitative reconstructions at a fixed validation index (zero-fill, real U-Net, complex U-Net, ground truth) with PSNR/SSIM overlaid.}
\label{fig:qualitative}
\end{figure}

\begin{figure}[H]
\centering
\includegraphics[width=0.8\linewidth]{figures/fig_training_curve.png}
\caption{Training curves showing per-step losses (left) and validation PSNR per epoch (right) for real and complex U-Nets.}
\label{fig:training_curve}
\end{figure}

\subsection{Scope differences and limitations}
This reproduction is narrower than \citet{cole2020analysisdeepcomplexvaluedconvolutional}: I use single-coil fastMRI knees only (no multi-coil or multi-dataset setting), equispaced masks at a fixed $R=6$/ACS=20 instead of variable-density Poisson masks across accelerations, and ~50k steps on 8k/1k slices with a single seed. I also restrict the comparison to U-Nets (no unrolled networks or activation-function ablations) and report magnitude PSNR/SSIM/$\ell_1$ without phase-specific metrics. These choices likely explain why the complex model’s gain is modest relative to the larger improvements reported in the original paper. Future work includes extending to multi-coil data, broader acceleration ranges, phase-sensitive metrics, and multiple seeds/architectures.
